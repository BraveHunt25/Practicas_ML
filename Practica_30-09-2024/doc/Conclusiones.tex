\section[Conclusiones]{Conclusiones}

\subsection[SPF]{Samuel Patiño Flores}
Con esta práctica vimos lo útil que es utilizar los métodos de reducción de dimensionalidad, como PCA y SOM. Esto nos ayuda a visualizar los datos de una forma más intuitiva, en forma de plano, y mostrando la información de manera más clara. En el caso de SOM, al capturar relaciones no lineales, lo hace más versátil en aplicaciones para clasificación.

\subsection[HJEY]{Hernández Jiménez Erick Yael}
Con ambos métodos, si bien se demostró que no se puede llegar a representar los datos originales con una fidelidad total, se puede decir que se llega a una representación considerablemente buena de los datos en $n$ dimensiones y, en el proceso, a una reducción conveniente para su análisis para posteriores procesos.  Cabe mencionar que, por los resultados, SOM es más conveniente para clasificar o reducir dimensiones de datos de manera no lineal y PCA para datos con relaciones más lineales. Además, dado que PCA no está hecho para clasificar, SOM es mejor para este objetivo aún cuando PCA pueda "accidentalmente" cumplir el cometido

\subsection[RGA]{Robert Garayzar Arturo}
La práctica de reducción de dimensionalidad con técnicas como PCA y SOM resulta sumamente útil para simplificar la representación de los datos, permitiendo un análisis más eficiente y menos propenso a sobreajustes. Si bien ambas técnicas tienen fortalezas particulares, PCA sobresale en la conservación de la estructura lineal de los datos, lo que lo convierte en una opción ideal para contextos donde las relaciones son claras y directas. Por otro lado, SOM permite capturar patrones más complejos y no lineales, abriendo la puerta a análisis más sofisticados en escenarios de clasificación. En resumen, la elección entre estas dos técnicas depende directamente de la naturaleza de los datos y el tipo de relaciones que se pretenda analizar.